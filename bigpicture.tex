\documentclass[12pt]{article}

\usepackage[hmargin=2cm,vmargin=2cm]{geometry}
% \usepackage{fullpage}
\usepackage{amsmath}
\usepackage{amssymb}
\usepackage{graphicx}
\usepackage{natbib}

\newcommand{\ul}{\underline}

\begin{document}
\section*{Goals}
improve SDSS calibration\\
\ul{mechanical instability}\\
seek to constrain instabilities in SDSS calibration especially to improve removal of pesky sky lines coinciding with science\\
current method normalizes each observation to sky/dome conditions of previous (accounting for short term instability)\\
current method ignores conditions at previous observations (failing to account for long term instability)\\
instability may be caused by some small number of mechanical factors that could be identified\\
develop model and use, say PCA, to remove effects\\
\ul{sky subtraction}\\
science data calibrated from flats (dome shots) and sky for that shot\\
idea: after using sky and flat to calibrate science, use science to validate weather\\
how? establish slit position/wavelength dependence on focal plane from science data\\
by co-adding lots of data, science cancels out, left with sky map to be used for better sky subtraction\\
would probably need to account for mechanical instability first\\
also want to know how big a difference both of these make
\end{document}

\section*{}
\subsection*{}
\subsubsection*{}

\begin{tabular}{c}
\end{tabular}

\begin{align*}
\end{align*}

\includegraphics[width=8cm,height=6cm]{121214fig.eps}

\begin{enumerate}
\item 
\end{enumerate}

\begin{eqnarray*}

\left\{\begin{array}{c}\end{array}\right\}

\begin{itemize}
\item 
\end{itemize}